\documentclass[11pt]{article} %Kopf mit Siegel  o h n e  Absenderleiste
\usepackage[latin1]{inputenc}
\usepackage{a4wide}

\usepackage{amssymb}
\usepackage{amsmath}
\usepackage{amsthm}
\usepackage{amsfonts}
\usepackage{enumitem}
\usepackage{multicol}
\usepackage{tikz,tikz-cd}
\usepackage{ifthen}     % um if-Abfrage zu machen
\usepackage{xfrac}      % for \sfrac
\usepackage{mathtools}  % for \coloneqq
\usepackage[normalem]{ulem}
\usepackage[pdftitle={Problem set Algebraic Geometry 3 Winter 2022/23},pdfauthor={Luca Battistella},pdfborder={0 0 0}]{hyperref}
\usepackage{todonotes}

\pagestyle{empty}
%
% Institutsbriefkopf des mathematischen Institutes
% Erstellt:  05.06.97 Jan Becker
%

\newlength{\Uwd}
\newlength{\Awd}
\font\ka=cmcsc10  scaled 1095
\font\kb=cmr12    scaled 1095
\font\kc=cmr10    scaled 1000
\font\ke=cmss10   scaled 1000
%\font\ks=cmhdseal scaled 1000
\font\kh=cmr12    scaled 1150
\font\ku=cmbx12   scaled 1728
\catcode`@=11

\def\Kopf#1#2#3#4#5#6{
\hbox{
 \settowidth{\Uwd}{\ku \quad Universit\"at Heidelberg\quad\ } %\quad
 \settowidth{\Awd}{\kc Im Neuenheimer Feld 288}

 \vspace*{7truemm}
 \raisebox{-20.3mm}{\ks A}
 {\kc\hspace{.2em}}
 \parbox[t]{\Uwd}{\centering{RUPRECHT-KARLS-UNIVERSIT\"AT HEIDELBERG } \\[1.3ex]  %\ku
                  \centering {\sc Mathematisches Institut} \\[1.5ex]                       %\kh%1.5
                  \centering{\ka #1\\
                  \centering{\kc #2}} }
 \hfill
 \ \ \parbox[t]{\Awd}{
               }
}
}

\newcommand{\uebungsblatt}[2]{
\mbox{}\vspace{-2.8cm}

\hspace{-2cm}
%\hbox{\Kopf{\small Prof.\ Dr.\ O. Venjakob}{}{5697}{venjakob}{mathi.uni-heidelberg.de}}
%\hbox{\Kopf{\small Prof.\ Dr.\ E. Freitag}{}{5762}{freitag}{mathi.uni-heidelberg.de}}
%\hbox{\Kopf{\small Prof.\ Dr.\ K. Wingberg}{}{4897}{wingberg}{mathi.uni-heidelberg.de}}
%\hbox{\Kopf{\small Dr.\ R. Busam}{}{5759}{busam}{mathi.uni-heidelberg.de}}
%\hbox{\Kopf{\small Der Dekan}{}{5758}{dekanat}{mathi.uni-heidelberg.de}}
%\hbox{\Kopf{\tiny Forschergruppe ``Arithmethik''}{}{5685}{}{mathi.uni-heidelberg.de}}
%\hbox{\Kopf{\small B. Schmoetten-Jonas}{}{5767}{schmoett}{mathi.uni-heidelberg.de}}
%\hbox{\Kopf{\ }{}{5767}{schmoett}{mathi.uni-heidelberg.de}{5697}}
%\hbox{\Kopf{\ }{}{5767}{schmoett}{mathi.uni-heidelberg.de}{5697}}

%\\[1.5truecm]

\vspace*{-10mm}
%
\mbox{}\hfill Luca Battistella, battistella@math.uni-frankfurt.de

\mbox{}\hfill Arne Kuhrs, kuhrs@math.uni-frankfurt.de

%\mbox{}\hfill \textit{Heidelberg, \heute}
\mbox{}\hfill Date out: \textit{#1}, Date in: \textit{#2}
\par\bigskip\bigskip\noindent
}

\newcommand{\intestazione}[1]{
\mbox{}\vspace{-2.8cm}

\hspace{-2cm}
%\hbox{\Kopf{\small Prof.\ Dr.\ O. Venjakob}{}{5697}{venjakob}{mathi.uni-heidelberg.de}}
%\hbox{\Kopf{\small Prof.\ Dr.\ E. Freitag}{}{5762}{freitag}{mathi.uni-heidelberg.de}}
%\hbox{\Kopf{\small Prof.\ Dr.\ K. Wingberg}{}{4897}{wingberg}{mathi.uni-heidelberg.de}}
%\hbox{\Kopf{\small Dr.\ R. Busam}{}{5759}{busam}{mathi.uni-heidelberg.de}}
%\hbox{\Kopf{\small Der Dekan}{}{5758}{dekanat}{mathi.uni-heidelberg.de}}
%\hbox{\Kopf{\tiny Forschergruppe ``Arithmethik''}{}{5685}{}{mathi.uni-heidelberg.de}}
%\hbox{\Kopf{\small B. Schmoetten-Jonas}{}{5767}{schmoett}{mathi.uni-heidelberg.de}}
%\hbox{\Kopf{\ }{}{5767}{schmoett}{mathi.uni-heidelberg.de}{5697}}
\hbox{\Kopf{\ }{}{5767}{schmoett}{mathi.uni-heidelberg.de}{5697}}

%\\[1.5truecm]

\vspace*{-6mm}
%
\mbox{}\hfill Vorlesung Differentialgeometrie II

%\mbox{}\hfill \textit{Heidelberg, \heute}
\mbox{}\hfill \textit{Heidelberg, #1}
\bn\bn
}

\newcommand{\titolo}[3]{

\vspace{-1.4cm}

\begin{center}
\textsc{\"Ubungsblatt #1}
\end{center}
\vspace{-0.2cm}
\begin{center}
\textbf{#2}
\end{center}
\vspace{-0.2cm}
\begin{center}
\textit{#3}\\
\end{center}
\rule{\textwidth}{0.1mm}

}

\newcommand{\examtitle}{

\vspace{-1.4cm}

\begin{center}
\textsc{Klausur}
\end{center}
\vspace{-0.2cm}
%\center{\textbf{#2}}
%\vspace{-0.2cm}
%\center{\textit{#3}}\\
\rule{\textwidth}{0.1mm}

}


\newcommand{\mockexamtitle}{

\vspace{-1.4cm}

\begin{center}
\textsc{Probeklausur}
\end{center}
\vspace{-0.2cm}
%\center{\textbf{#2}}
%\vspace{-0.2cm}
%\center{\textit{#3}}\\
\rule{\textwidth}{0.1mm}

}

\newcommand{\secondexamtitle}{

\vspace{-1.4cm}

\begin{center}
\textsc{Nachklausur}
\end{center}
\vspace{-0.2cm}
%\center{\textbf{#2}}
%\vspace{-0.2cm}
%\center{\textit{#3}}\\
\rule{\textwidth}{0.1mm}

}



\theoremstyle{definition}
\newtheorem{exercise}{Exercise}
\newtheorem{aufgabe}{Aufgabe}

\catcode`@=12
\setlength{\topmargin}{-15pt}
\endinput



\textwidth=16.1truecm     %habe ich eingef_gt (1.3.00)
\parindent0mm


\newcounter{problem}
\setcounter{problem}{1}

\newcommand{\problem}[2][]{\bigskip\medskip {\textbf{Problem \theproblem}} \ {\em #1} {\ifthenelse{\equal{e}{#2}}{}{\\[1mm]}
} \stepcounter{problem}}

\newcommand{\bonusproblem}[1][]{\bigskip\medskip {\textbf{Bonus problem \theproblem}} {\em #1}  {(not to be handed in)\\[1mm]} \stepcounter{problem}}

\long\def\symbolfootnote[#1]#2{\begingroup
\def\thefootnote{\fnsymbol{footnote}}\footnote[#1]{#2}\endgroup}

\newcommand{\issuedate}{Nov 01, 2022}
\newcommand{\returndate}{--}
\newcommand{\sheetnumber}{3}
\newcommand{\keywords}{Ring spectra, sheaves, Jacobson rings}


\newcommand{\HH}{{\mathbb{H}}}
\newcommand{\RR}{{\mathbb{R}}}
\newcommand{\CC}{{\mathbb{C}}}
\newcommand{\ZZ}{{\mathbb{Z}}}
\newcommand{\PP}{\mathbb P}
\newcommand{\Aaff}{\mathbb A}
\newcommand{\on}{\operatorname}
\newcommand{\Spec}{\operatorname{Spec}}


\newcommand{\bcd}{\begin{center}\begin{tikzcd}}
\newcommand{\ecd}{\end{tikzcd}\end{center}}

\DeclareMathOperator{\Hom}{Hom}

%\addtolength{\textwidth}{2.2cm}
%\addtolength{\hoffset}{-1.1cm}
%\addtolength{\textheight}{3.4cm}
%\addtolength{\voffset}{-1.7cm}


\begin{document}

\uebungsblatt{\issuedate}{\returndate}

\vspace{-0.6cm}
\begin{center}
{\LARGE\textsc{Algebraic geometry 1}}

\textsc{problem set \sheetnumber}
\end{center}
\vspace{-0.2cm}
\begin{center}
\textbf{Keywords:} \keywords
\end{center}
\vspace{-0.2cm}
\rule{\textwidth}{0.1mm}

\problem{}
Let $k$ be a field, and let $K=k(x,y)$ be the field of rational functions in two variables.

Let $G=\mathbb Z^2$ with the lexicographic order ($(a,b)\leq(c,d)$ if and only if $a<c$, or $a=c$ and $b\leq d$).

\begin{enumerate}
 \item Let $v(x^ay^b)=(a,b)$, $v(\sum c_{a,b}x^ay^b)=\min\{(a,b)|c_{a,b}\neq 0\}$, and $v(\frac{f}{g})=v(f)-v(g)$. Show that $v$ defines a valuation on $K$ with value group $G$.
 \item Show that the valuation ring $R=\{f\in K|v(f)\geq 0\}$ is not Noetherian.
 \item  Show that $\Gamma=\{c\in G|c\geq0\}$ is a submonoid of $G$. \emph{Ideals} of $\Gamma$ are subsets $I$ of $\Gamma$ such that for all $\alpha\in I,\gamma\in\Gamma$ we have $\alpha+\gamma\in I$. %of the form $I_{a,b}=\{c\in G|c\geq(a,b)\},(a,b)\in\Gamma$.
 An ideal $I$ is called \emph{prime} if $c_1+c_2\in I\Rightarrow c_1\in I$ or $c_2\in I$. Show that (prime) ideals of $R$ are in bijection with (prime) ideals of $\Gamma$.
 
 Describe the topological space $\on{Spec}(R)$.
\end{enumerate}
 

\problem{}
Let $X$ be a scheme. Show that points of $X$ are in bijection with equivalence classes of morphisms from fields spectra $f\colon\on{\Spec}(F)\to X$, where $f_1\sim f_2$ if there is a common field extension $\iota_i\colon F_i\hookrightarrow \Omega,\ i=1,2$ such that the following diagram is commutative:
\bcd
\on{Spec}(\Omega)\ar[r,"\iota_1^\#"]\ar[d,"\iota_2^\#"] & \Spec(F_1)\ar[d,"f_1"] \\
\Spec(F_2)\ar[r,"f_2"]& X
\ecd

\problem{}
Let $f\colon X\to Y$ be a continuous map of topological spaces, let $\mathcal{G}$ be a presheaf of Abelian groups on $Y$. Consider the following association:
\[U\subseteq_{\text{open}}X\mapsto f^{-1}G(U)=\underrightarrow{lim}_{f(U)\subseteq V\subseteq_{\text{open}}Y}\mathcal G(V).\]
Show that there are canonically defined restriction homomorphisms making this into a presheaf of Abelian groups on $X$. Show that the following adjunction property holds: for every presheaf of Abelian groups $\mathcal{F}$ on $X$, one has
\[\on{Hom}_{(\text{PreSh}_X)}(f^{-1}\mathcal{G},\mathcal{F})=\on{Hom}_{(\text{PreSh}_Y)}(\mathcal{G},f_*\mathcal{F}).\]
Notice that, when $f$ is the inclusion of a point $p$ in $Y$, $f^{-1}\mathcal{G}=\mathcal{G}_p$ (the stalk of $\mathcal{G}$ at $p$).

In general, with notations as above, for every $p\in X$ we have
\[(f^{-1}\mathcal{G})_p=\mathcal{G}_{f(p)}.\]

Show that in general $f^{-1}\mathcal{G}$ is not a sheaf even if $\mathcal{G}$ is.

\problem{}
Let $R$ be a ring (commutative with $1$). Show that the following are equivalent:
\begin{enumerate}
 \item every prime ideal is intersection of maximals;
 \item every radical ideal is intersection of maximals;
 \item $V_m(I)=V_m(J)$ (where $V_m(I)=V(I)\cap\Spec_m(R)$) implies $\sqrt{I}=\sqrt{J}$;
 \item $\Spec_m(R)\subseteq\Spec(R)$ is dense in every closed subset of $\Spec(R)$;
 \item the association $Z\mapsto Z\cap \Spec_m(R)$ induces a bijection between the closed subsets of $\Spec(R)$ and those of $\Spec_m(R)$.
\end{enumerate}
A ring satisfying (any one of) the above properties is called \emph{Jacobson}.

Show that a local ring is Jacobson if and only if it has Krull dimension $0$.

\end{document}
